\section{Background}

\subsection{Choice of database and methods}
We have downloaded the \textit{Enqu\^ete emploi en continu 2015} database from the INSEE website.
\textit{Enqu\^ete Emploi} is one of the central elements of the statistical system of knowledge of
employment and unemployment. It is the only source for implementing the definition of unemployment
within the meaning of the International Labor Office. The survey provides original data on
occupations, female and youth activity, hours of work, precarious employment and wages. It makes it
possible to better understand the situation of the unemployed as well as changes in the situation
with regard to work: change from school to work, retirement activity, changes in occupation.
\textit{Enqu\^ete Emploi} contains detailed data on training.

% TODO: add more stuff here
% TODO: ajouter un graphique histogramme (1 seule barre) avec la valeur actop montrant
% d'une couleur les 0 et d'une autre couleur 1 pour mettre en avant que 65% des gens sont actifs

\subsection{Explanation on econometrical methods}
\subsubsection{Linear Regression}

Suppose that we have $n$ distinct predictors. Then the multiple linear regression model takes the
form:

\begin{equation}
    y = \beta_0 + \beta_1 x_1 + \beta_2 x_2 + ... + \beta_n x_n + \epsilon
\end{equation}

where $x_i$ represents the i$^{th}$ predictor and $\beta_i$ quantifies the association between that
variable and the response. We interpret $\beta_i$ as the average effect on $y$ of a one-unit
increase in $x_i$, holding all other predictors fixed.

The regression coefficients $\beta_0, \beta_1, ..., \beta_n$ are unknown, and must be estimated.
Given estimates $\hat{\beta_0}, \hat{\beta_1}, ..., \hat{\beta_n}$, we can make predictions using
the formula:

\begin{equation}
    \hat{y} = \hat{\beta_0} + \hat{\beta_1} x_1 + \hat{\beta_2} x_2 + ··· + \hat{\beta_n} x_n
\end{equation}

The parameters are estimated using the same least squares approach, that is, we choose $\beta_0,
\beta_1, ..., \beta_n$ to minimize the sum of squared residuals.

Once we have fit the multiple regression model, it is straightforward to predict the response $y$ on
the basis of a set of values for the predictors $x_1, x_2, ..., x_n$. However, there are three sorts
of uncertainty associated with this prediction:

\begin{enumerate}
    \item The coefficient estimates $\hat{\beta_0}, \hat{\beta_1}, ..., \hat{\beta_n}$ are estimates
for $\beta_0, \beta_1, ..., \beta_n$. That is, the least squares plane $\hat{y} = \hat{\beta_0} +
\hat{\beta_1} x_1 + ··· + \hat{\beta_n} x_n$ is only an estimate for the true population regression
plane $f(x) = \beta_0 + \beta_1 x_1 + ··· + \beta_n x_n$. The inaccuracy in the coefficient
estimates is related to the reducible error. We can compute a confidence interval in order to
determine how close $\hat{y}$ will be to $y = f(x)$.

    \item Of course, in practice assuming a linear model for $f(x)$ is almost always an
approximation of reality, so there is an additional source of potentially reducible error which we
call model bias. So when we use a linear model, we are in fact estimating the best linear
approximation to the true surface. However, here we will ignore this discrepancy, and operate as if
the linear model were correct.

    \item Even if we knew $f(x)$ --- that is, even if we knew the true values for $\beta_0, \beta_1,
..., \beta_n$ --- the response value cannot be predicted perfectly because of the random error
$\epsilon$. How much will $y$ vary from $\hat{y}$? We use prediction intervals to answer this
question. Prediction intervals are always wider than confidence intervals, because they incorporate
both the error in the estimate for $f(x)$ (the reducible error) and the uncertainty as to how much
an individual point will differ from the population regression plane (the irreducible error).
\end{enumerate}


\subsubsection{Classification, the logit and probit models}
The linear regression model assumes that the response variable Y is quantitative. But in many
situations, the response variable is instead qualitative. For example, eye color is qualitative,
taking on values blue, brown, or green. Often qualitative variables are referred to as categorical ;
we will use these terms interchangeably. In this chapter, we study approaches for predicting
qualitative responses, a process that is known as classification. Predicting a qualitative response
for an observation can be referred to as classifying that observation, since it involves assigning
the observation to a category, or class. On the other hand, often the methods used for
classification first predict the probability of each of the categories of a qualitative variable, as
the basis for making the classification. In this sense they also behave like regression methods.
There are many possible classification techniques, or classifiers, that one might use to predict a
qualitative response. In this chapter we two of the most widely-used classifiers: logistic and
probit regression.

For an OLS model, we have:

\begin{equation}
    y = x'\beta + \epsilon
\end{equation}

Binary outcome models estimate the probability that $y=1$ as a function of the independent
variables:

\begin{equation}
    p = P(y=1 | x) = F(x’\beta)
\end{equation}

There are 3 different models depending on the functional form of $F(x’\beta)$:

\begin{enumerate}
    \item Regression model (linear probability model): $F(x'\beta) = x'\beta$
    \item Logit model: $F(x'\beta)$ is the cdf of the logistic distribution
    \begin{equation}
        F(x'\beta) = \Lambda(x'\beta) = \frac{e^{x'\beta}}{1+e^{x'\beta}}
    \end{equation}
    The predicted probabilities are limited between 0 and 1.
    \item Probit model: $F(x'\beta)$ is the cdf of the standard normal distribution
    \begin{equation}
        F(x'\beta) = \Phi(x'\beta) = \int_{-\infty}^{+\infty} \phi(z)dz
    \end{equation}
    The predicted probabilities are also limited between 0 and 1.
\end{enumerate}

% TODO: graphs of the linear, logit and probit models

Probit and logit models are estimated using the maximum likelihood method. An increase in $x$
increases/decreases the likelihood that $y=1$ (makes that outcome more/less likely). In other words,
an increase in $x$ makes the outcome of $1$ more or less likely. We interpret the sign of the
coefficient but not the magnitude. The magnitude cannot be interpreted using the coefficient because
different models have different scales of coefficients.

Coefficients differ among models because of the functional form of the $F$ function:

$$\beta_{logit} \approx 4\beta_{OLS}$$
$$\beta_{probit} \approx 2.5\beta_{OLS}$$
$$\beta_{logit} \approx 1.6\beta_{logit}$$

% TODO: explain where this comes from

Now let’s look at the marginal effects. For the OLS regression model, the marginal effects are the
coefficients and they do not depend on $x$. For the logit and probit models, the marginal effects
are calculated as:

\begin{equation}
    \frac{\partial p}{\partial x_j} = F'(x'\beta) \times \beta_j
\end{equation}

The marginal effects depend on $x$, so we need to estimate the marginal effects at a specific value
of $x$ (typically the means).

\paragraph{Marginal effects for the logit model}

\begin{equation}
    \frac{\partial p}{\partial x_j} = \beta_j\Lambda(x'\beta)(1-\Lambda(x'\beta)) =
    \\beta_j\frac{e^{x'\beta}}{(1+e^{x'\beta})^2}
\end{equation}

\paragraph{Marginal effects for the probit model}
\begin{equation}
    \frac{\partial p}{\partial x_j} = \beta_j\Phi'(x'\beta) = \beta_j\phi'(x'\beta)
\end{equation}

\paragraph{Interpretation of marginal effects}
\begin{itemize}
    \item An increase in $x$ increases (decreases) the probability that $y=1$ by the marginal effect
expressed as a percentage
    \begin{itemize}
        \item For dummy independent variables, the marginal effect is expressed in comparison to the
    base category ($x=0$).
        \item For continuous independent variables, the marginal effect is expressed for a one-unit
    change in $x$.
    \end{itemize}
    \item We interpret both the sign and the magnitude of the marginal effects.
    \item The probit and logit models produce almost identical marginal effects.

\end{itemize}

\paragraph{Odds ratios/relative risk for the logit model}
The odds ratio or relative risk is $\frac{p}{1-p}$ and measures the probability that $y=1$ relative
to the probability that $y=0$.

$$p = \frac{e^{x'\beta}}{1+e^{x'\beta}}$$
$$\frac{p}{1-p} = e^{x'\beta}$$
$$\ln \frac{p}{1-p} = x'\beta$$

An odds ratio of $2$ means that the outcome $y=1$ is twice as likely as the outcome of $y=0$. Odds
ratios are estimated with the logistic model.


\subsection{Machine Learning methods}
\subsubsection{Decision trees}

In the following, we describe tree-based methods for regression and classification. These involve
stratifying or segmenting the predictor space into a number of simple regions. In order to make a
prediction for a given observation, we typically use the mean or the mode of the training
observations in the region to which it belongs. Since the set of splitting rules used to segment the
predictor space can be summarized in a tree, these types of approaches are known as decision tree
methods. Decision trees can be applied to both regression and classification problems. We first
consider regression problems, and then move on to classification.

\paragraph{Regression trees}
We discuss the process of building a regression tree. There are two steps:
\begin{enumerate}
    \item We divide the predictor space --- that is, the set of possible values for $x_1, x_2, ...,
x_n$ --- into $J$ distinct and non-overlapping regions $R_1, R_2, ..., R_J$.
    \item For every observation that falls into the region $R_j$, we make the same prediction, which
is simply the mean of the response values for the training observations in $R_j$.
\end{enumerate}

We now elaborate on Step 1 above. How do we construct the regions $R_1, ..., R_J$? In theory, the
regions could have any shape. However, we choose to divide the predictor space into high-dimensional
rectangles, or boxes, for simplicity and for ease of interpretation of the resulting predictive
model. The goal is to find boxes $R_1, ..., R_J$ that minimize the RSS, given by:

$$\sum_{j=1}^{J} \sum_{i \in R_j} (y_i - \hat{y}_{R_j})^2$$

where $\hat{y}_{R_j}$ is the mean response for the training observations within the $j^{th}$ box.

The process may produce good predictions on the training set, but is likely to overfit the data,
leading to poor test set performance. This is because the resulting tree might be too complex. A
smaller tree with fewer splits (that is, fewer regions $R_1, ..., R_J$) might lead to lower variance
and better interpretation at the cost of a little bias. One possible alternative to the process
described above is to build the tree only so long as the decrease in the RSS due to each split
exceeds some (high) threshold. This strategy will result in smaller trees, but is too short-sighted
since a seemingly worthless split early on in the tree might be followed by a very good split — that
is, a split that leads to a large reduction in RSS later on. Therefore, a better strategy is to grow
a very large tree $T_0$, and then prune it back in order to obtain a subtree.

\paragraph{Classification trees}
A classification tree is very similar to a regression tree, except that it is used to predict a
qualitative response rather than a quantitative one. Recall that for a regression tree, the
predicted response for an observation is given by the mean response of the training observations
that belong to the same terminal node. In contrast, for a classification tree, we predict that each
observation belongs to the most commonly occurring class of training observations in the region to
which it belongs. In interpreting the results of a classification tree, we are often interested not
only in the class prediction corresponding to a particular terminal node region, but also in the
class proportions among the training observations that fall into that region. The task of growing a
classification tree is quite similar to the task of growing a regression tree. Just as in the
regression setting, we use recursive binary splitting to grow a classification tree. However, in the
classification setting, RSS cannot be used as a criterion for making the binary splits. A natural
alternative to RSS is the classification error rate. Since we plan to assign an observation in a
given region to the most commonly occurring class of training observations in that region, the
classification error rate is simply the fraction of the training observations in that region that do
not belong to the most common class:

$$E = 1 - \max_k{\hat{p}_{mk}}$$

Here, $\hat{p}_{mk}$ represents the proportion of training observations in the mth region that are
from the $k_{th}$ class. However, it turns out that classification error is not sufficiently
sensitive for tree-growing, and in practice two other measures are preferable. The Gini index is
defined by:

$$G = \sum_{k=1}^{K} \hat{p}_{mk}(1-\hat{p}_{mk})$$

a measure of total variance across the $K$ classes. It is not hard to see that the Gini index takes
on a small value if all of the $\hat{p}_{mk}$’s are close to zero or one. For this reason the Gini
index is referred to as a measure of node purity --- a small value indicates that a node contains
predominantly observations from a single class. When building a classification tree, the Gini index
is typically used to evaluate the quality of a particular split. It might be used when pruning the
tree, but the classification error rate is preferable if prediction accuracy of the final pruned
tree is the goal.

\paragraph{Advantages and drawbacks of trees}
First, the advantages:
\begin{itemize}
    \item Trees are very easy to explain to people. In fact, they are even easier to explain than
linear regression.
    \item Some people believe that decision trees more closely mirror human decision-making than do
the regression and classification approaches ;
    \item Trees can be displayed graphically, and are easily interpreted even by a non-expert
(especially if they are small).
    \item Trees can easily handle qualitative predictors without the need to create dummy variables.
\end{itemize}

But, unfortunately, trees generally do not have the same level of predictive accuracy as some of
other regression and classification approaches. However, by aggregating many decision trees, using
methods like bagging, random forests, and boosting, the predictive performance of trees can be
substantially improved. We introduce these concepts in the next paragraph.

\subsubsection{Bagging, random forests, boosting}

\paragraph{Bagging}
The decision trees suffer from high variance. This means that if we split the training data into two
parts at random, and fit a decision tree to both halves, the results that we get could be quite
different. In contrast, a procedure with low variance will yield similar results if applied
repeatedly to distinct data sets; linear regression tends to have low variance, if the ratio of $n$
to $p$ is moderately large. Bootstrap aggregation, or \textit{bagging}, is a general-purpose
procedure for reducing the variance of a statistical learning method; we introduce it here because
it is particularly useful and frequently used in the context of decision trees. Recall that given a
set of $n$ independent observations $Z_1$, ..., $Z_n$, each with variance $\sigma^2$, the variance
of the mean of $Z$ of the observations is given by $\frac{\sigma^2}{n}$. In other words, averaging a
set of observations reduces variance. Hence a natural way to reduce the variance and hence increase
the prediction accuracy of a statistical learning method is to take many training sets from the
population, build a separate prediction model using each training set, and average the resulting
predictions. In other words, we could calculate $\hat{f}_1(x), \hat{f}_2(x), ..., \hat{f}_B(x)$
using $B$ separate training sets, and average them in order to obtain a single low-variance
statistical learning model, given by:

$$\hat{f}_{avg}(x) = \frac{1}{B} \sum_{b=1}^{B} \hat{f}^b(x)$$

Of course, this is not practical because we generally do not have access to multiple training sets.
Instead, we can bootstrap, by taking repeated samples from the (single) training data set. In this
approach we generate $B$ different bootstrapped training data sets. We then train our method on the
$b^{th}$ bootstrapped training set in order to get $\hat{f}^b(x)$, and finally average all the
predictions, to obtain:

$$\hat{f}_{bag}(x) = \frac{1}{B} \sum_{b=1}^{B} \hat{f^*}^b(x)$$

This is called bagging.

While bagging can improve predictions for many regression methods, it is particularly useful for
decision trees. To apply bagging to regression trees, we simply construct $B$ regression trees using
$B$ bootstrapped training sets, and average the resulting predictions. These trees are grown deep,
and are not pruned. Hence each individual tree has high variance, but low bias. Averaging these $B$
trees reduces the variance. Bagging has been demonstrated to give impressive improvements in
accuracy by combining together hundreds or even thousands of trees into a single procedure.

\paragraph{Random Forests}
Random forests provide an improvement over bagged trees by way of a small tweak that decorrelates
the trees. As in bagging, we build a number of decision trees on bootstrapped training samples. But
when building these decision trees, each time a split in a tree is considered, a random sample of
$m$ predictors is chosen as split candidates from the full set of $p$ predictors. The split is
allowed to use only one of those $m$ predictors. A fresh sample of $m$ predictors is taken at each
split, and typically we choose $m \approx \sqrt{p}$ --- that is, the number of predictors considered
at each split is approximately equal to the square root of the total number of predictors. In other
words, in building a random forest, at each split in the tree, the algorithm is not even allowed to
consider a majority of the available predictors. This may sound crazy, but it has a clever
rationale. Suppose that there is one very strong predictor in the data set, along with a number of
other moderately strong predictors. Then in the collection of bagged trees, most or all of the trees
will use this strong predictor in the top split. Consequently, all of the bagged trees will look
quite similar to each other. Hence the predictions from the bagged trees will be highly correlated.
Unfortunately, averaging many highly correlated quantities does not lead to as large of a reduction
in variance as averaging many uncorrelated quantities. In particular, this means that bagging will
not lead to a substantial reduction in variance over a single tree in this setting. Random forests
overcome this problem by forcing each split to consider only a subset of the predictors. Therefore,
on average $\frac{p-m}{p}$ of the splits will not even consider the strong predictor, and so other
predictors will have more of a chance. We can think of this process as decorrelating the trees,
thereby making the average of the resulting trees less variable and hence more reliable.

\paragraph{Boosting}
We now discuss boosting, yet another approach for improving the predictions resulting from a
decision tree. Like bagging, boosting is a general approach that can be applied to many statistical
learning methods for regression or classification. Here we restrict our discussion of boosting to
the context of decision trees. Recall that bagging involves creating multiple copies of the original
training data set using the bootstrap, fitting a separate decision tree to each copy, and then
combining all of the trees in order to create a single predictive model. Notably, each tree is built
on a bootstrap data set, independent of the other trees. Boosting works in a similar way, except
that the trees are grown sequentially: each tree is grown using information from previously grown
trees. Boosting does not involve bootstrap sampling; instead each tree is fit on a modified version
of the original data set.

Consider first the regression setting. Like bagging, boosting involves combining
a large number of decision trees, $\hat{f}_1, ...,  \hat{f}_B$.

\begin{verbatim}
1. Set ˆ f(x) = 0 and ri = yi for all i in the training set.
2. For b = 1, 2, . . . ,B, repeat:
(a) Fit a tree ˆ fb with d splits (d+1 terminal nodes) to the training
data (X, r).
(b) Update ˆ f by adding in a shrunken version of the new tree:
ˆ f(x) ← ˆ f(x) + λ ˆ fb(x).
(c) Update the residuals,
ri ← ri − λ ˆ fb(xi).
3. Output the boosted model,
\end{verbatim}

Output: $\hat{f}(x) = \frac{1}{B} \sum_{b=1}^{B} \lambda \hat{f}^b (x)$

% TODO: fix the above

What is the idea behind this procedure? Unlike fitting a single large decision tree to the data,
which amounts to fitting the data hard and potentially overfitting, the boosting approach instead
learns slowly. Given the current model, we fit a decision tree to the residuals from the model. That
is, we fit a tree using the current residuals, rather than the outcome $y$, as the response. We then
add this new decision tree into the fitted function in order to update the residuals. Each of these
trees can be rather small, with just a few terminal nodes, determined by the parameter $d$ in the
algorithm. By fitting small trees to the residuals, we slowly improve $\hat{f}$ in areas where it
does not perform well. The shrinkage parameter $\lambda$ slows the process down even further,
allowing more and different shaped trees to attack the residuals. In general, statistical learning
approaches that learn slowly tend to perform well. Note that in boosting, unlike in bagging, the
construction of each tree depends strongly on the trees that have already been grown. We have just
described the process of boosting regression trees. Boosting classification trees proceeds in a
similar but slightly more complex way, and the details are omitted here. Boosting has three tuning
parameters:

\begin{enumerate}
    \item The number of trees $B$. Unlike bagging and random forests, boosting can overfit if $B$ is
too large, although this overfitting tends to occur slowly if at all. We use cross-validation to
select $B$.
    \item The shrinkage parameter $\lambda$, a small positive number. This controls the rate at
which boosting learns. Typical values are $0.01$ or $0.001$, and the right choice can depend on
the problem. Very small $\lambda$ can require using a very large value of $B$ in order to
achieve good performance.
    \item The number $d$ of splits in each tree, which controls the complexity of the boosted
ensemble. Often $d=1$ works well, in which case each tree is a stump, consisting of a single split.
In this case, the boosted ensemble is fitting an additive model, since each term involves only a
single variable. More generally, $d$ is the interaction depth, and controls the interaction order of
the boosted model, since $d$ splits can involve at most $d$ variables.
\end{enumerate}
