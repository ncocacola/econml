\section{Current progress} \subsection{Research} On the advice of our supervisor, we have researched
the theoretical background of the project. More specifically, we have looked into (and learned)
basic Econometrics methods that we will need to apply throughout the project, using
\cite{wooldridge}, and in conjunction with the Econometrics I class taught by F. Kramarz. We have
also started learning some Machine Learning methods, understanding the basics using online material
\cite{vidhya}, and then studying further with the help of a Machine Learning textbook \cite{introR}.

\subsection{Setup} As this project will involve a fair amount of programming, we have also ensured
that we have an adequate experimental setup that will allow us to implement the algorithms and
experiment in a relatively flexible way, yet also let us record and analyse our results with
relative ease.

Our experimental setup revolves around various tools that we have put in place:
\begin{itemize}
    \item the \texttt{Python} and \texttt{R} programming languages, along with the
\texttt{scikit-learn} machine learning package \cite{python}\cite{R}\cite{learn}
    \item the \texttt{Git} source control manager \cite{git}, which enables us to keep track of
different versions of our code (and thus project), while also letting us easily rollback changes,
or run multiple versions of the code (e.g. with different parameters)
    \item \texttt{Jupyter} \cite{jupyter}, an interactive computing platform, which enables us to
run code directly in the browser, and mix text with code snippets to document our work properly
\end{itemize}

We have also downloaded the dataset we intend to work with, \emph{Enqu\^ete Emploi en Continu}
\cite{enquete}, performed some basic manipulations to make it easier to work with (conversions), and
started investigating the kinds of data we have access to, as well as running some very basic
regressions (more for practice than for finding useful results).

Our work can be found at all times on the project's
\href{https://github.com/ncocacola/econml/}{GitHub repository} and in the
\href{https://github.com/ncocacola/econml/blob/master/econml.ipynb}{Jupyter Notebook}.
