\section{Machine Learning}
\textit{This section relies heavily on the theoretical background provided by \cite{introR}, and applications in Python found in \cite{mlPython}.}


\subsection{Decision Trees}

Simple decision trees for regression and classification involve segmenting the predictor space into a number of simple regions. Then, in order to make a
prediction for a given observation, we typically use the mean or the mode of the training observations in the region to which it belongs. Since the set of splits used to segment the space can be summarised in a tree, these approaches are known as \textit{decision tree methods}.

\subsubsection{Regression trees}
The process of building a regression tree typically involves two steps:

\begin{enumerate}
    \item First, we divide the predictor space -- that is, the set of possible values for $x_1, x_2, ...,
x_n$ -- into $J$ distinct and non-overlapping regions $R_1, R_2, ..., R_J$.
    \item For every observation that falls into the region $R_j$, we predict the response value as the mean of the response values for the training observations in $R_j$.
\end{enumerate}

\paragraph{How are the regions $R_1, ..., R_J$ constructed?}
In theory, the regions $R_1, ..., R_J$ could have any shape, but in practice, the predictor space is typically divided into high-dimensional
rectangles, or \textit{boxes} for simplicity and for ease of interpretation of the resulting model. The goal is then to find boxes $R_1, ..., R_J$ that minimise the RSS, given by:

\begin{equation}
    \sum_{j=1}^{J} \sum_{i \in R_j} (y_i - \hat{y}_{R_j})^2
\end{equation}

where $\hat{y}_{R_j}$ is the mean response for the training observations within the $j^{th}$ box.

While this method may produce good predictions on the training set, it is likely to overfit the data because the resulting tree might be too complex. A
smaller tree with fewer splits (that is, fewer regions $R_1, ..., R_J$) might lead to lower variance and better interpretation at the cost of a little bias.

There are two strategies to make the resulting tree simpler:

\begin{enumerate}
    \item Build the tree only so long as the decrease in the RSS due to each split exceeds some threshold; this method is considered somewhat "short-sighted" as an early below-threshold split might be followed by a very good split, i.e. a split that leads to a large reduction in RSS
    \item Grow a very large tree $T_0$, and then prune it back in order to obtain a simpler subtree
\end{enumerate}

\subsubsection{Classification trees}
The distinction between regression and classification trees is very similar to the one between the linear and logit/probit regressions. As we have seen in the previous section, in a regression tree, the predicted response for an observation is given by the (mean) response of the training observations that belong to the same region.

In contract, the classification tree predicts that each observation belongs to the \textit{most commonly occuring class} among training observations in the region to which it belongs.

Within this setting, the RSS criterion cannot be used for making the binary splits, so the \textit{classification error rate} (i.e. the fraction of the training observations in that region that do not belong to the most common class) can be used instead:

\begin{equation}
    E = 1 - \max_k{\hat{p}_{mk}}
\end{equation}

where $\hat{p}_{mk}$ represents the proportion of training observations in the $m^{th}$ region that are from the $k_{th}$ class.

However, it turns out that classification error is not sufficiently sensitive for tree-growing, and in practice the \textit{Gini index} is preferred:

\begin{equation}
    G = \sum_{k=1}^{K} \hat{p}_{mk}(1-\hat{p}_{mk})
\end{equation}

The Gini index is a measure of total variance across the $K$ classes, and takes on a small value if all of the $\hat{p}_{mk}$’s are close to zero or one. As such, it is also sometimes referred to as a measure of \textit{node purity}, where a small value indicates that a node mostly contains observations from a single class.

% TODO: isn't entropy missing here?

\subsubsection{Advantages \& drawbacks}
\begin{itemize}
    \item Trees are very easy to explain to people, even easier than linear regression.
    \item Some people believe that decision trees more closely mirror human decision-making than do the regression and classification approaches ;
    \item Trees can be displayed graphically, and are easily interpreted even by a non-expert (especially if they are small).
    \item Trees can easily handle qualitative predictors without the need to create dummy variables.
\end{itemize}

However, trees usually do not have the same level of accuracy as some regression and classification approaches (as we will see later in the \textit{Results} section).


\subsection{Bagging, random forests, boosting}

Simple trees typically suffer from high variance --- that is, the results of a decision tree fit on distinct data sets are likely to be very different. In constrast, methods such as linear regression tend to have low variance. A popular and substantial improvement over simple trees is the aggregation of many such trees, using methods like bagging, random forests, and boosting.

\subsubsection{Bagging}

Bootstrap aggregation, or \textit{bagging}, is a procedure for reducing the variance of a statistical learning method. Recall that given a set of $n$ independent observations $Z_1$, ..., $Z_n$, each with variance $\sigma^2$, the variance of the mean of $Z$ of the observations is given by $\frac{\sigma^2}{n}$. In other words, averaging a set of observations reduces variance.

Hence, a natural way to reduce the variance and increase the prediction accuracy of a statistical learning method is to take many training sets from the population, build a separate prediction model using each training set, and average the resulting predictions. In other words, we calculate $\hat{f}_1(x), \hat{f}_2(x), ..., \hat{f}_B(x)$ using $B$ separate training sets, and average them in order to obtain a single low-variance statistical learning model, given by:

$$\hat{f}_{avg}(x) = \frac{1}{B} \sum_{b=1}^{B} \hat{f}^b(x)$$

In practice, we do not have access to multiple training sets, but we can use a procedure called \textit{boostrap} to generate such sets, by taking repeated samples from the (single) training data set. In this approach we generate $B$ different bootstrapped training data sets. We then train our method on the $b^{th}$ bootstrapped training set in order to get $\hat{f}^b(x)$, and finally average all the predictions, to obtain:

$$\hat{f}_{bag}(x) = \frac{1}{B} \sum_{b=1}^{B} \hat{f^*}^b(x)$$

Bagging has been demonstrated to give impressive improvements in accuracy by combining together hundreds or even thousands of trees into a single procedure.

\subsubsection{Random Forests}
Random forests provide an improvement over bagging by way of a small tweak that "decorrelates" the trees. As in bagging, we build a number of decision trees on samples bootstrapped from the original dataset (i.e. created from random sampling with replacement).

The only distinction between bagging and random forests is that the latter selects, at each candidate split, a random subset of the features. The reason for doing this is that, in bagging, if some of the features are very strong predictors for the output, these features will be selected in many of the splits. As a consequence, all of the bagged trees will look quite similar to each other and the predictions from the bagged trees will be highly correlated. In particular, this means that bagging will not lead to a reduction in variance over a simple tree.

Random forests overcome this problem by forcing every split to pick predictors from a random subset, thus decorrelating the trees and reducing the variance of the resulting trees.

\subsubsection{Boosting}
Recall that bagging involves creating multiple copies of the original dataset using the "boostrap" method, fitting a separate tree to each copy, and then combining all of the trees to create a single predictive model.

Boosting does something similar, but instead of building each tree on top of a boostrapped dataset (independently of other trees), the trees a grown sequentially, using information from previous trees.

Note that in boosting, unlike in bagging, the construction of each tree depends strongly on the trees that have already been grown. From a computational point of view, this means that it is impossible to calculate boosting-based predictions in parallel, thus considerably slowing down the computation.

Boosting has three tuning parameters:

\begin{enumerate}
    \item The number of trees $B$. Unlike bagging and random forests, boosting can overfit if $B$ is too large, although this overfitting tends to occur slowly if at all. We use cross-validation to select $B$.
    \item The shrinkage parameter $\lambda$, a small positive number. This controls the rate at which boosting learns. Typical values are $0.01$ or $0.001$, and the right choice can depend on the problem. Very small $\lambda$ can require using a very large value of $B$ in order to achieve good performance.
    \item The number $d$ of splits in each tree, which controls the complexity of the boosted ensemble. Often $d=1$ works well, in which case each tree is a stump, consisting of a single split. In this case, the boosted ensemble is fitting an additive model, since each term involves only a single variable. More generally, $d$ is the interaction depth, and controls the interaction order of the boosted model, since $d$ splits can involve at most $d$ variables.
\end{enumerate}
